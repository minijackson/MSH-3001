\documentclass[12pt]{article}\usepackage[]{graphicx}\usepackage[]{color}
%% maxwidth is the original width if it is less than linewidth
%% otherwise use linewidth (to make sure the graphics do not exceed the margin)
\makeatletter
\def\maxwidth{ %
  \ifdim\Gin@nat@width>\linewidth
    \linewidth
  \else
    \Gin@nat@width
  \fi
}
\makeatother

\definecolor{fgcolor}{rgb}{0.345, 0.345, 0.345}
\newcommand{\hlnum}[1]{\textcolor[rgb]{0.686,0.059,0.569}{#1}}%
\newcommand{\hlstr}[1]{\textcolor[rgb]{0.192,0.494,0.8}{#1}}%
\newcommand{\hlcom}[1]{\textcolor[rgb]{0.678,0.584,0.686}{\textit{#1}}}%
\newcommand{\hlopt}[1]{\textcolor[rgb]{0,0,0}{#1}}%
\newcommand{\hlstd}[1]{\textcolor[rgb]{0.345,0.345,0.345}{#1}}%
\newcommand{\hlkwa}[1]{\textcolor[rgb]{0.161,0.373,0.58}{\textbf{#1}}}%
\newcommand{\hlkwb}[1]{\textcolor[rgb]{0.69,0.353,0.396}{#1}}%
\newcommand{\hlkwc}[1]{\textcolor[rgb]{0.333,0.667,0.333}{#1}}%
\newcommand{\hlkwd}[1]{\textcolor[rgb]{0.737,0.353,0.396}{\textbf{#1}}}%

\usepackage{framed}
\makeatletter
\newenvironment{kframe}{%
 \def\at@end@of@kframe{}%
 \ifinner\ifhmode%
  \def\at@end@of@kframe{\end{minipage}}%
  \begin{minipage}{\columnwidth}%
 \fi\fi%
 \def\FrameCommand##1{\hskip\@totalleftmargin \hskip-\fboxsep
 \colorbox{shadecolor}{##1}\hskip-\fboxsep
     % There is no \\@totalrightmargin, so:
     \hskip-\linewidth \hskip-\@totalleftmargin \hskip\columnwidth}%
 \MakeFramed {\advance\hsize-\width
   \@totalleftmargin\z@ \linewidth\hsize
   \@setminipage}}%
 {\par\unskip\endMakeFramed%
 \at@end@of@kframe}
\makeatother

\definecolor{shadecolor}{rgb}{.97, .97, .97}
\definecolor{messagecolor}{rgb}{0, 0, 0}
\definecolor{warningcolor}{rgb}{1, 0, 1}
\definecolor{errorcolor}{rgb}{1, 0, 0}
\newenvironment{knitrout}{}{} % an empty environment to be redefined in TeX

\usepackage{alltt}

\usepackage[utf8]{inputenc}
\usepackage[T1]{fontenc}
\usepackage[french]{babel}
\usepackage[hidelinks]{hyperref}
\usepackage{amsmath}
%\usepackage{fullpage}
%\usepackage{microtype}
\usepackage[backend=biber]{biblatex}
\usepackage{float}
\usepackage{csquotes}

\title{Étude de marché pour les nouvelles gammes PSA Peugeot-Citroën}
\author{Marc \textsc{Autrand},Timothée \textsc{Pallot}, Edouard
	\textsc{Nguon}, Rémi \textsc{Nicole}}
\date{11 décembre 2015}

\addbibresource{liography.bib}
\IfFileExists{upquote.sty}{\usepackage{upquote}}{}
\begin{document}

\maketitle

\break
~
\newpage
\break
\tableofcontents

\part{Pré-enquête}

\section{Introduction}

\paragraph{} Dans le cadre de l'unité MSH-3001, nous avons réalisé une étude
sur les marques DS et Citroën, notamment la C4 Cactus. Carlos Tavarez nous a
demandé de le conseiller dans sa stratégie marketing.

\paragraph{} Nous avons donc sondé une centaine de personnes afin de connaître
leur avis sur la C4 Cactus: son design, ses fonctionnalités, la stratégie de
Citroën en créant une voiture totalement différente de sa production
habituelle. Nous avons aussi obtenu leurs avis sur la marque DS, son design
comparé à des marques de luxe comme Audi et les conséquences de la séparation
avec la marque Citroën sur cette dernière.

\paragraph{} Nous avons aussi réussi à discuter avec deux conseillers
commerciaux de Renault et Peugeot, contrairement à des concessionnaires comme
Citroën et Toyota qui n'ont pas trouvé le temps de nous
répondre du fait de la forte présence de clients.

\break
\section{Historique}

\paragraph{} L'histoire récente de PSA Peugeot-Citroën est marquée par
plusieurs bouleversements. Depuis le rachat de Citroën par Michelin en 1976, le
groupe doit relever le défi de faire cohabiter deux marques concurrentes. Si la
distinction claire entre les clients Citroën et Peugeot, creusée depuis les
années 1950, semblait être un avantage, le groupe parvint difficilement à
concilier les différences de culture et d'approche des deux constructeurs. Peu
à peu, il s'éloigne du marché haut de gamme, duquel Citroën était pourtant
proche.

\paragraph{} En 2009, le lancement de la gamme DS ouvre de nouveaux horizons.
Créé par le PDG sortant Christian Streiff, le label mise sur le retour très
populaire des anciens modèles phares des marques européennes : Mini Cooper,
Fiat 500, etc. Si le sigle DS apporte l'héritage de la gamme mythique des
années 1960, le design est celui d'une marque premium, tout comme le prix des
modèles. Citroën part ainsi à l'assaut de ténors comme Audi, mais aussi du
marché chinois, longtemps délaissé, pour lequel le prix a moins d'importance
que la qualité du véhicule.

\paragraph{} La ligne DS est un succès. Pour autant, sa communication l'éloigne
de Citroën ; puis, en 2014, le groupe PSA choisit d'en faire une filiale à part
entière. Ne reste alors plus à Citroën que sa seconde spécialité, celle de
faire du différent. Elle présente ainsi au même moment un modèle d'un genre
nouveau, entre la berline et le SUV, et doté d'un design particulier : la C4
Cactus, première d'une génération de voitures vouée à redéfinir la gamme
Citroën.

\paragraph{} Si la stratégie du groupe PSA depuis 2014 a été efficace, en
renouant dès le premier semestre 2015 avec les bénéfices\cite{lesechos}, le
futur est plus incertain. La C4 Cactus est un modèle clivant qui rassemble de
nombreux fans mais souffre d'un manque de reconnaissance du grand public. La
gamme DS, quant à elle, prend son envol en Chine et devient le porte-étendard
du groupe.\\

\noindent \emph{Quel développement peuvent connaître ces deux gammes, et
	peuvent-elles coexister?}

\break
\section{Méthodologie}

\paragraph{} Afin de réaliser notre étude, nous sommes allés rencontrer des
concessionnaires comme Peugeot et Renault afin d'avoir l'avis personnel de
conseillers commerciaux. Pour élargir notre champ de recherche, nous avons
aussi créé un sondage qui nous avons fait passer dans des réseaux sociaux comme
Facebook afin d'avoir des avis divers et variés venant de catégories de
personnes différentes.

\paragraph{} Le positionnement de la Cactus, proche des SUV, lui donnant plus
de chances de trouver un public à l'étranger qu'en France, nous avons également
cherché des retours à l'international. Cette décision était confortée par les
prix reçus par le véhicule : voiture de l'année en Espagne et au Danemark, prix
du design de New York, etc. Nous avons ainsi réalisé une version du
sondage en langue anglaise, afin de la partager sur les réseaux sociaux mais
aussi auprès de communautés d'amateurs de voitures françaises. Si une majorité
de retours nous sont venus du Royaume-Uni, des réponses ont aussi été données
depuis le Brésil, la Turquie et plusieurs pays européens.\\

\noindent Le total de personnes interrogées s'élève à plus de 150.

\part{Résultats}

\section{Retour des échanges avec les concessionnaires}

\paragraph{} Nous avons pu interroger deux conseillers commerciaux des
concessionnaires Peugeot et Renault, le premier étant partenaire de Citroën dans
le groupe PSA, et le second un concurrent de la marque.

\paragraph{} Le conseiller de Renault nous a donné un avis plutôt négatif
concernant la C4 Cactus. Il travaille à Renault, mais ne se considère pas comme
fidèle à Renault, ce qui nous a permis d'avoir un avis plutôt subjectif de la
part d'un homme ayant certainement la vingtaine. Il n'est pas du tout satisfait
du design de la C4 Cactus, trouvant les air-bumps disgracieux et pas efficaces.
En effet cette personne a travaillé dans la production des voitures avant de
s'engager dans le commercial et a vu beaucoup de Citroën C4 Cactus revenir au
garage à cause des air-bumps car lorsque ceux-ci sont abîmés et doivent être
changés, il faut changer les deux portes étant reliées à l'air-bump cassé et
tout le plastique autour, ce qui demande beaucoup de temps et de main d'œuvre.
De plus, de son point de vue, la partie avant de la voiture ressemble trop à
celle d'un $4 \times 4$, elle est trop imposante ce qui ne correspond pas à ce que
produit Citroën habituellement.

\paragraph{} En ce qui concerne la marque DS, son avis est totalement
différent. Pour lui, le design des voitures de la marque est très beau
esthétiquement et la nouvelle marque a eu la bonne idée de se séparer de
Citroën. Cependant, les prix affichés par DS sont trop élevés pour prétendre
atteindre des ventes aussi importantes que ses concurrents.

\paragraph{} Le conseiller commercial de Peugeot que nous avons interrogé a lui
aussi un avis très négatif sur la C4 Cactus. Le design de cette voiture est
beaucoup trop décalée par rapport à ce que fabrique généralement la marque. La
technologie air-bump, notamment, n'est pas du tout du goût de conseiller qui ne
la trouve absolument pas exceptionnelle malgré tout le bourrage médiatique fait
par la marque autour de celle-ci.

\paragraph{} Pour ce qui est de la marque DS, elle a fabriqué de très bons
produits comme la DS3 et la DS4, mais celui-ci se rapproche beaucoup trop de la
Citroën C4: si on ne regarde pas les phares, les deux voitures sont identiques.
Pour lui, la marque DS a plusieurs gros problèmes: le premier est que les
premières DS et notamment les plus médiatisées DS3 et DS4 ont été vendues par
Citroën, et non par DS. Le deuxième problème de la marque est sa politique de
prix: elle fournit des voitures à prix élevés de l'ordre de ceux des véhicules
produits par Audi ou BMW, mais pas de la même qualité.

\paragraph{} De plus, Citroën a voulu aller trop vite car à l'origine, la
marque produisait des voitures qui n'étaient pas adaptées au public des jeunes
personnes.  Elle ne produit pas non plus de voitures de luxe, et n'était pas
assez reconnue pour avoir l'effet escompté sur le grand public. Elle a voulu
être en avance sur son temps, mais s'est essoufflée. Par exemple, pour la
Peugeot 3008, les ingénieurs de la marque ont conçus cette voiture en 2009,
mais la voiture n'a été fabriquée que 4 ans plus tard. Ils ont eu la présence
d'esprit d'avoir anticipé le futur besoin des gens en avance de 4 ans et de la
commercialiser au bon moment.

\paragraph{} Enfin, le dernier problème de DS se situe dans ses garages: ils
n'ont pas misé leurs efforts sur la qualité, ce qui n'est pas respectueux de
la part d'un constructeur de voitures de luxe dont les prix sont de l'ordre de
50 000 voire 70 000 €.

\section{État des ventes de la Cactus et de DS, publics concernés}

\subsection{Citroën C4 Cactus}

\paragraph{} Avec sa mise en vente en 2014, la C4 Cactus connaît un début
discret en termes de vente mais a réussi tout de même à enregistrer 75 000
ventes au bout d'un an ainsi que 35 distinctions à travers le
monde\cite{communique}, mais seulement 27\% ont été vendues en France. Les
Français sont beaucoup moins attirés par ce modèle, car le modèle n'a pas été
commandé plus de 20 000 fois depuis son lancement en Juin 2014 et a été vendus
en seulement 6 000 exemplaires en cette année 2015. Il faut aussi prendre en
compte le fait que les agents de la marque en ont presque tous une par
obligation et les concessionnaires en ont en stock donc ces chiffres ne sont
pas tout à fait révélateurs du nombre de clients qui en ont acheté une. De
plus, le modèle ne pointe qu'à la 22\ieme{} place des voitures les plus vendues
en France depuis le début de l'année 2015.

\paragraph{} Ce modèle allie confort et technologie utile et vise un public
cherchant plus un voiture utile qu'un voiture permettant d'affirmer son statut
social. En termes de prix, la Citroën Cactus se positionne entre voitures ``low
cost'' et voitures haut de gamme. En effet, son prix peut varier entre 15 200 €
et 22 900 € à neuf selon la finition et le moteur pris\cite{banquette}.

\subsection{Citroën DS}

\paragraph{} La marque DS s'est véritablement lancée avec la DS3 en 2010 qui a
été vendue en 26 417 exemplaires cette même année, 33 015 l'année suivante,
puis les ventes ont diminué avec 25 107 ventes en 2012 et 19 672 et 17 471
respectivement en 2013 et 2014. Ce modèle a plu à de nombreux clients de part
son design novateur\cite{avis}, son comportement routier exceptionnel, son
confort mais aussi son intérieur original et moderne. Les finitions sont
satisfaisantes, la voiture est très personnalisable, a une bonne habitabilité à
l'arrière et n'est pas très chère par rapport à ce que fait déjà Citroën,
puisqu'ils s'articulent autour de 20 000 € pour les modèles les plus chers.

\paragraph{} En revanche, elle a quelques défauts comme l'équipement qui n'est
pas très développé, la partie plastique qui n'est pas d'une grande qualités et
des bruits parasites viennent de temps en temps agacer le conducteur. Le
dernier point noir de ce modèle se situe au niveau des vitres arrières qui ne
peuvent pas s'ouvrir. La DS4 vient s'ajouter au marché en 2011, soit un an après
la sortie de son aînée et elle ne suscite pas autant d'admiration chez les
clients, car les ventes n'arrivent pas à décoller : 10 757 voitures vendues lors
de la première année, pour atteindre le pic de ventes l'année d'après avec 14
143 modèles vendus et retomber à 11 740 et 8 661 ventes en 2013 et 2014.

\paragraph{} La DS5 fait même moins de ventes que la DS4 puisque le nombre
maximum de voitures vendues sur une année est de 10 943 et concerne l'année de
sa sortie. Les deux année suivantes sont très mauvaises, pour ne pas dire
catastrophiques car les ventes tombent à 8 365 en 2013 et 5 614 en 2014. En
effet les défauts de la DS3 ne sont pas résolus sauf en ce qui concerne
l'équipement de la voiture. Le confort global est lui beaucoup moins bon sur la
DS4 et la DS5 que sur la DS3.

\paragraph{} Cette nouvelle marque se distingue beaucoup de Citroën dans son
design, son luxe mais aussi son prix. Elle ne touche plus le citoyen lambda,
mais se rapproche plus des marques allemandes telles que Audi et BMW. Elle
touche donc plus les jeunes, qui sont attirés par son style, son design, ses
couleurs.

\break
\part{Analyse des résultats du sondage}

\section{Âge et sexe}



\begin{knitrout}
\definecolor{shadecolor}{rgb}{0.969, 0.969, 0.969}\color{fgcolor}\begin{figure}[H]
\includegraphics[width=\maxwidth]{figure/tranche_age_fr-1} \caption[Tranches d'âges des personnes ayant répondu au sondage français]{Tranches d'âges des personnes ayant répondu au sondage français}\label{fig:tranche age fr}
\end{figure}


\end{knitrout}
\paragraph{} Du fait que nous ayons envoyé notre sondage français sur les
réseaux sociaux, nous avons récolté la plupart de nos réponses de la part des
jeunes dont la tranche d'âge est de 15-30 ans.

\begin{knitrout}
\definecolor{shadecolor}{rgb}{0.969, 0.969, 0.969}\color{fgcolor}\begin{figure}[H]
\includegraphics[width=\maxwidth]{figure/tranche_age_en-1} \caption[Tranches d'âges des personnes ayant répondu au sondage anglais]{Tranches d'âges des personnes ayant répondu au sondage anglais}\label{fig:tranche age en}
\end{figure}


\end{knitrout}

\paragraph{} En ce qui concerne notre sondage anglais, que nous avons posté sur
la page Facebook anglaise de Citroën, les résultats sont beaucoup plus diverses
et concerne surtout plus les personnes plus âgées, de plus de 40 ans. Elles
représentent 79\% des personnes ayant répondu à cette enquête.

\begin{knitrout}
\definecolor{shadecolor}{rgb}{0.969, 0.969, 0.969}\color{fgcolor}\begin{figure}[H]
\includegraphics[width=\maxwidth]{figure/sexe_fr-1} \caption[Sexe des personnes ayant répondu au sondage français]{Sexe des personnes ayant répondu au sondage français}\label{fig:sexe fr}
\end{figure}


\end{knitrout}

\begin{knitrout}
\definecolor{shadecolor}{rgb}{0.969, 0.969, 0.969}\color{fgcolor}\begin{figure}[H]
\includegraphics[width=\maxwidth]{figure/sexe_en-1} \caption[Sexe des personnes ayant répondu au sondage anglais]{Sexe des personnes ayant répondu au sondage anglais}\label{fig:sexe en}
\end{figure}


\end{knitrout}

\paragraph{} Parmi toutes les personnes ayant répondu à notre sondage,
seulement 32 sur 156 sont des femmes, soit 21\%, ce qui fait assez peu. Ceci
montre que les femmes sont beaucoup moins intéressées par les voitures que les
hommes, en particulier ce qui concerne celles produites par Citroën.

\break
\section{Marketing}

\begin{knitrout}
\definecolor{shadecolor}{rgb}{0.969, 0.969, 0.969}\color{fgcolor}\begin{figure}[H]
\includegraphics[width=\maxwidth]{figure/know_fr-1} \caption[Proportion des personnes connaissant auparavant la C4 Cactus ayant répondu au sondage français]{Proportion des personnes connaissant auparavant la C4 Cactus ayant répondu au sondage français}\label{fig:know fr}
\end{figure}


\end{knitrout}

\begin{knitrout}
\definecolor{shadecolor}{rgb}{0.969, 0.969, 0.969}\color{fgcolor}\begin{figure}[H]
\includegraphics[width=\maxwidth]{figure/know_en-1} \caption[Proportion des personnes connaissant auparavant la C4 Cactus ayant répondu au sondage anglais]{Proportion des personnes connaissant auparavant la C4 Cactus ayant répondu au sondage anglais}\label{fig:know en}
\end{figure}


\end{knitrout}

\paragraph{} Pour ce qui est de la connaissance de la C4 Cactus, les deux
sondages donnent à peu près les mêmes résultats : les étrangers ayant répondu
au sondage anglais connaissent tous ce modèle, tandis qu'il n'est pas connu
unanimement en France : 15\% des personnes sondées ne le connaissent pas.



\begin{knitrout}
\definecolor{shadecolor}{rgb}{0.969, 0.969, 0.969}\color{fgcolor}\begin{figure}[H]
\includegraphics[width=\maxwidth]{figure/means_fr-1} \caption[Nuage de mots du moyen de prise de conscience de l'existence de la C4 Cactus des personnes ayant répondu au sondage français]{Nuage de mots du moyen de prise de conscience de l'existence de la C4 Cactus des personnes ayant répondu au sondage français}\label{fig:means fr}
\end{figure}


\end{knitrout}

\begin{knitrout}
\definecolor{shadecolor}{rgb}{0.969, 0.969, 0.969}\color{fgcolor}\begin{figure}[H]
\includegraphics[width=\maxwidth]{figure/means_en-1} \caption[Nuage de mots du moyen de prise de conscience de l'existence de la C4 Cactus des personnes ayant répondu au sondage anglais]{Nuage de mots du moyen de prise de conscience de l'existence de la C4 Cactus des personnes ayant répondu au sondage anglais}\label{fig:means en}
\end{figure}


\end{knitrout}

\paragraph{} Ce nuage de point nous démontre que la majorité des gens
connaissent la C4 Cactus grâce aux publicités, aux magasines et à Internet. En
ce qui concerne le sondage anglais, du fait qu'il ait été publié sur un site de
Citroën, une grande partie des personnes ont répondu qu'elles connaissent la C4
Cactus car elles suivent l'actualité de la marque.

\break
\section{Apparence de la voiture}

\begin{knitrout}
\definecolor{shadecolor}{rgb}{0.969, 0.969, 0.969}\color{fgcolor}\begin{figure}[H]
\includegraphics[width=\maxwidth]{figure/design_fr-1} \caption[Nuage de mots de l'avis du design de la C4 Cactus des personnes ayant répondu au sondage français]{Nuage de mots de l'avis du design de la C4 Cactus des personnes ayant répondu au sondage français}\label{fig:design fr}
\end{figure}


\end{knitrout}

\begin{knitrout}
\definecolor{shadecolor}{rgb}{0.969, 0.969, 0.969}\color{fgcolor}\begin{figure}[H]
\includegraphics[width=\maxwidth]{figure/design_en-1} \caption[Nuage de mots de l'avis du design de la C4 Cactus des personnes ayant répondu au sondage anglais]{Nuage de mots de l'avis du design de la C4 Cactus des personnes ayant répondu au sondage anglais}\label{fig:design en}
\end{figure}


\end{knitrout}

\paragraph{} Les avis sur le design de la C4 Cactus sont assez partagés chez
les Français : on retrouve le plus souvent les termes ``très original'',
``design'', ``esthétique'', ``beau'', mais aussi ``moche''. Les anglais sont
eux, contre toute attente, très flatteurs vis-à-vis du modèle.

\begin{knitrout}
\definecolor{shadecolor}{rgb}{0.969, 0.969, 0.969}\color{fgcolor}\begin{figure}[H]
\includegraphics[width=\maxwidth]{figure/interior_fr-1} \caption[Nuage de mots de l'avis de l'intérieur de la C4 Cactus des personnes ayant répondu au sondage français]{Nuage de mots de l'avis de l'intérieur de la C4 Cactus des personnes ayant répondu au sondage français}\label{fig:interior fr}
\end{figure}


\end{knitrout}

\begin{knitrout}
\definecolor{shadecolor}{rgb}{0.969, 0.969, 0.969}\color{fgcolor}\begin{figure}[H]
\includegraphics[width=\maxwidth]{figure/interior_en-1} \caption[Nuage de mots de l'avis de l'intérieur de la C4 Cactus des personnes ayant répondu au sondage anglais]{Nuage de mots de l'avis de l'intérieur de la C4 Cactus des personnes ayant répondu au sondage anglais}\label{fig:interior en}
\end{figure}


\end{knitrout}

\paragraph{} Concernant l'intérieur de la voiture, les termes ``bien'',
``bonne'', mais aussi ``simple'', ``great'', ``like'' reviennent très souvent,
qui montre que l'intérieur du modèle est très apprécié par les clients, non
seulement les loyaux anglais, mais aussi plusieurs français dont l'attrait pour
la voiture n'est pas unanime.

\break
\section{Opinion générale}



\begin{knitrout}
\definecolor{shadecolor}{rgb}{0.969, 0.969, 0.969}\color{fgcolor}\begin{figure}[H]
\includegraphics[width=\maxwidth]{figure/qualities_fr-1} \caption[Qualités de la C4 Cactus selon les personnes ayant répondu au sondage français]{Qualités de la C4 Cactus selon les personnes ayant répondu au sondage français}\label{fig:qualities fr}
\end{figure}


\end{knitrout}

\begin{knitrout}
\definecolor{shadecolor}{rgb}{0.969, 0.969, 0.969}\color{fgcolor}\begin{figure}[H]
\includegraphics[width=\maxwidth]{figure/flaws_fr-1} \caption[Défauts de la C4 Cactus selon les personnes ayant répondu au sondage français]{Défauts de la C4 Cactus selon les personnes ayant répondu au sondage français}\label{fig:flaws fr}
\end{figure}


\end{knitrout}

\paragraph{} Les français sont aussi très partagés dans les qualités et défauts
de la C4 Cactus. Plus de 20 personnes ont répondu que son originalité, son
design extérieur, la simplicité de son intérieur et de ses fonctionnalités, et
ses coussins ``airbump'' constituent ses qualités, alors que 24 personnes
pensent que son design extérieur, comprenant les airbumps, constitue son
principal défaut. L'incertitude sur le public visé et le décalage avec l'esprit
Citroën constituent les deux autres principaux défauts remarqués par les
français.

\begin{knitrout}
\definecolor{shadecolor}{rgb}{0.969, 0.969, 0.969}\color{fgcolor}\begin{figure}[H]
\includegraphics[width=\maxwidth]{figure/qualities_en-1} \caption[Qualités de la C4 Cactus selon les personnes ayant répondu au sondage anglais]{Qualités de la C4 Cactus selon les personnes ayant répondu au sondage anglais}\label{fig:qualities en}
\end{figure}


\end{knitrout}

\begin{knitrout}
\definecolor{shadecolor}{rgb}{0.969, 0.969, 0.969}\color{fgcolor}\begin{figure}[H]
\includegraphics[width=\maxwidth]{figure/flaws_en-1} \caption[Défauts de la C4 Cactus selon les personnes ayant répondu au sondage anglais]{Défauts de la C4 Cactus selon les personnes ayant répondu au sondage anglais}\label{fig:flaws en}
\end{figure}


\end{knitrout}

\paragraph{} Pour les étrangers actifs sur la page Facebook de Citroën, la
Citroën a plusieurs qualités qui sont son originalité, son design, son
comportement routier, la simplicité de son intérieur et de ses fonctionnalités
mais aussi ses airbumps. Cependant, 5 personnes trouvent que le public visé par
la marque est incertain, et 6 personnes trouvent que le nombre de
fonctionnalités est trop faible.

\begin{knitrout}
\definecolor{shadecolor}{rgb}{0.969, 0.969, 0.969}\color{fgcolor}\begin{figure}[H]
\includegraphics[width=\maxwidth]{figure/buy_fr-1} \caption[Proportion personnes ayant répondu au sondage français qui achèteraient une C4 Cactus]{Proportion personnes ayant répondu au sondage français qui achèteraient une C4 Cactus}\label{fig:buy fr}
\end{figure}


\end{knitrout}

\begin{knitrout}
\definecolor{shadecolor}{rgb}{0.969, 0.969, 0.969}\color{fgcolor}\begin{figure}[H]
\includegraphics[width=\maxwidth]{figure/buy_en-1} \caption[Proportion personnes ayant répondu au sondage anglais qui achèteraient une C4 Cactus]{Proportion personnes ayant répondu au sondage anglais qui achèteraient une C4 Cactus}\label{fig:buy en}
\end{figure}


\end{knitrout}

\paragraph{} Pour finir avec la Citroën C4 Cactus, un quart des Français
interrogés achèteraient ce modèle s'ils devaient acheter une automobile, alors
que 94\% des étrangers sondés feraient aussi ce choix.

\break
\section{La marque DS}

\begin{knitrout}
\definecolor{shadecolor}{rgb}{0.969, 0.969, 0.969}\color{fgcolor}\begin{figure}[H]
\includegraphics[width=\maxwidth]{figure/ds_know_fr-1} \caption[Proportion personnes ayant répondu au sondage français sachant que DS était une marque indépendante]{Proportion personnes ayant répondu au sondage français sachant que DS était une marque indépendante}\label{fig:ds know fr}
\end{figure}


\end{knitrout}

\begin{knitrout}
\definecolor{shadecolor}{rgb}{0.969, 0.969, 0.969}\color{fgcolor}\begin{figure}[H]
\includegraphics[width=\maxwidth]{figure/ds_know_en-1} \caption[Proportion personnes ayant répondu au sondage anglais sachant que DS était une marque indépendante]{Proportion personnes ayant répondu au sondage anglais sachant que DS était une marque indépendante}\label{fig:ds know en}
\end{figure}


\end{knitrout}

\paragraph{} La part des Français sachant que DS est une marque indépendante et
très similaire à celle des étrangers : 57 et 62\% respectivement.

\begin{knitrout}
\definecolor{shadecolor}{rgb}{0.969, 0.969, 0.969}\color{fgcolor}\begin{figure}[H]
\includegraphics[width=\maxwidth]{figure/brand_fr-1} \caption[Nuage de mots de l'avis de la marque DS des personnes ayant répondu au sondage français]{Nuage de mots de l'avis de la marque DS des personnes ayant répondu au sondage français}\label{fig:brand fr}
\end{figure}


\end{knitrout}

\begin{knitrout}
\definecolor{shadecolor}{rgb}{0.969, 0.969, 0.969}\color{fgcolor}\begin{figure}[H]
\includegraphics[width=\maxwidth]{figure/brand_en-1} \caption[Nuage de mots de l'avis de la marque DS des personnes ayant répondu au sondage anglais]{Nuage de mots de l'avis de la marque DS des personnes ayant répondu au sondage anglais}\label{fig:brand en}
\end{figure}


\end{knitrout}

\paragraph{} Concernant les avis sur la marque DS, le français et étrangers
sont aussi d'accord sur ce point : cette nouvelle marque incarne le luxe et un
design novateur et beau, et est aussi très chère.

\begin{knitrout}
\definecolor{shadecolor}{rgb}{0.969, 0.969, 0.969}\color{fgcolor}\begin{figure}[H]
\includegraphics[width=\maxwidth]{figure/preference_fr-1} \caption[Préférences de marque des personnes ayant répondu au sondage français]{Préférences de marque des personnes ayant répondu au sondage français}\label{fig:preference fr}
\end{figure}


\end{knitrout}

\begin{knitrout}
\definecolor{shadecolor}{rgb}{0.969, 0.969, 0.969}\color{fgcolor}\begin{figure}[H]
\includegraphics[width=\maxwidth]{figure/preference_en-1} \caption[Préférences de marque des personnes ayant répondu au sondage anglais]{Préférences de marque des personnes ayant répondu au sondage anglais}\label{fig:preference en}
\end{figure}


\end{knitrout}

\paragraph{} Cependant, ils n'ont pas le même regard sur la concurrence
incarnée par Audi et BMW notamment, car si plus de la moitié des français sont
très attirés par la première marque allemande citée, les Anglais sont eux en
faveur de la marque française.

\paragraph{}

\part{Conclusion}

\paragraph{} Un an après son lancement, la Citroën C4 Cactus ne se hisse qu'à la 22ème place des voitures les plus vendues en France depuis le début de l’année 2015. Ce succès mitigé peut s'expliquer en France par l'originalité du véhicule, dont le design s'éloigne des sentiers battus de Citroën. Si notre enquête a révélé que ce modèle proche des SUV rassemble de nombreux amateurs, il semble qu'il soit moins adapté au public français, moins amateur de SUV et souvent rebuté par l'aspect extérieur du véhicule et de ses \textit{airbumps}. Son prix, son confort et ses fonctionnalités en font un produit davantage destiné à des acheteurs à la recherche d'un véhicule utilitaire. Si l'objectif de Citroën était de proposer aux automobilistes français un nouveau type de véhicule quotidien, la Cactus paraît plus apte à conquérir le marché européen.


\paragraph{} La marque DS, quant à elle, s’est lancée sur le segment \textit{premium}, une gamme au dessus de la maison-mère Citroën ; un positionnement assumé par le prix et le design des véhicules. Elle veut incarner la voiture de luxe à la française et ainsi concurrencer les ténors européens Audi et BMW. Les résultats de nos sondages montrent que le jeune public français (20-30 ans) a plus d'affinités pour Audi que pour DS et BMW ; toutefois, il ne semble pas départager ces deux dernières. Ce constat permet de saluer une première réussite de DS, qui a su en quelques années s'approcher du niveau de popularité des marques \textit{premium} historiques. La marque s'offre ainsi les moyens de rivaliser avec Audi, BMW ou encore Lexus dans les années à venir, en profitant du renouvellement de leur public.


\paragraph{} Toutefois, alors que le public français plébiscite la prise d'indépendance de la marque, voyant Citroën sur une toute autre gamme, les amateurs de voitures françaises à l'étranger s'étonnent davantage de ce choix. Citroën a en effet conservé, notamment au Royaume-Uni, l'image de marque dont il jouissait entre 1960 et 1990 et dont la DS fait partie intégrante. Le groupe PSA Peugeot-Citroën doit donc relever un défi de communication : affirmer DS tout en redéfinissant Citroën, l'une comme égérie du luxe à la française, l'autre comme moteur d'innovation.

\printbibliography%

\end{document}

% vim: spell : spelllang=fr
